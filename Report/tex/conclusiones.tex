\section{Conclusiones}
El proyecto concluye de manera exitosa.
La implementación de un modelo no lineal de péndulo simple y la construcción
de un modelo para validar los resultados producidos 
fue una experiencia bastante enriquecedora.
Esto debido al alcance total del proyecto: modelado,
simulación, construcción, calibración y comparación.\\

Cada etapa del proyecto requirió de habilidades diferentes, así como 
de la búsqueda de información adecuada para poder avanzar con el desarrollo.
La etapa de modelado hizo uso de las distintas formulaciones de mecánica 
que fueron presentadas durante el curso.
El desarrollo del simulador a su vez requirió de habilidades de programación
 y la capacidad de desarrollar una estrategia de trabajo que permitiera integrar
 los resultados de simulación con las áreas subsecuentes: documentación y construcción.\\

Mediante el uso de múltiples herramientas de medición fue posible calibrar el 
simulador para que el modelo virtual describa un movimiento mucho 
más cercano al comportamiento del péndulo real.
Además, la integración de nuevas herramientas de análisis permitió determinar 
propiedades del sistema que no eran posibles con los sensores disponibles.
El uso de Tracker demostró ser esencial para la obtención de los nuevos valores
del sistema debido a su capacidad de obtener detalles de la mecánica del sistema
que no fueron posibles medir con las herramientas disponibles.\\

Finalmente, el planteamiento del sistema bajo las distintas formulaciones 
de mecánica permiten obtener nuevas perspectivas sobre el modelado de sistemas.
Las formulaciones basadas en energía demuestran su poder ser adaptadas
de manera más fácil a cualquier conjunto de coordenadas.
Esta ventaja sobre la formulación de Newton resulta atractiva para modelos más complejos.
No obstante es esencial tomar en cuenta qué restricciones tienen estas formulaciones
y si es posible extenderlas para incluir los efectos de fuerzas no conservativas.
