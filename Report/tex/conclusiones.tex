\section{Conclusiones}

El desarrollo de un modelo matemático para el péndulo 
simple abre la puerta para la aplicación
de conocimientos de las áreas de física y matemática 
en un sistema de fácil construcción. 
Esto ha permitido desarrollar a mayor detalle el estudio
del péndulo simple, pues es posible realizar comparaciones
entre las simulaciones desarrolladas con el modelo matemático,
el análisis de video del mecanismo
y las mediciones tomadas por el mismo sistema.
Este ciclo de retroalimentación da lugar a un refinamiento del 
modelo, pues es posible determinar con mayor precisión los 
valores del sistema real e implementarlos en el simulador.

Se confirma la validez del modelo matemático para el caso de fricción.
Al analizar el comportamiento de $\theta$ en la simulación, 
como se muestra en la figura
\ref{fig: time plot theta dtheta friction},
se observa que esta gráfica describe el mismo
perfil de movimiento que el péndulo real (figura \ref{fig: mindstorms theta}).


