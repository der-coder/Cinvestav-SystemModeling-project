\section{Introducción}

%  What is a simple pendulum?

El péndulo simple (referido solamente como péndulo) es uno 
de los sistemas no lineares más estudiados por la 
comunidad científica y académica. La proliferación de su
uso como sistema base para evaluar estrategias de control ha
permitido desarrollar un conocimiento detallado del sistema.\\


De acuerdo a \cite{sastry}, para el modelo del péndulo simple, 
el péndulo se asume como una masa ($m$) concentrada en un extremo
de una barra. El otro extremo se encuentra restringido por una
junta de rotación que limita su movimiento a un plano. 
Se asume una fuerza de fricción viscosa ($F_f$) con factor de amortiguamiento
 ($k$). El péndulo oscila debido al efecto de la gravedad ($F_{mg}$)
Las figuras \ref{fig: simple pendulum} y \ref{fig: pendulum forces} exhiben 
los componentes antes mencionados.

 \begin{figure}[ht]
    \centering
    \import{./img/}{pendulum_diagram.pdf_tex}
    \caption{Sistema de Péndulo simple.}
    \label{fig: simple pendulum}
\end{figure}

%  What did we do?

Para describir el comportamiento del sistema es necesario desarrollar
ecuaciones para detallar el movimiento y velocidad del cuerpo $m$ 
respecto al marco referencial de interés.

