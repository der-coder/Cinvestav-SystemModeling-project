\section{Introducción}

El péndulo simple (referido solamente como péndulo) es uno 
de los sistemas no lineares más estudiados por la 
comunidad científica y académica. La proliferación de su
uso como sistema base para evaluar estrategias de control ha
permitido desarrollar un conocimiento detallado del sistema.


De acuerdo a \cite{sastry}, para el modelo del péndulo simple, 
el péndulo se asume como una masa $m$ concentrada en un extremo
de una barra. El otro extremo se encuentra restringido por una
junta de rotación que limita su movimiento a un plano. 
Se asume una fuerza de fricción viscosa $F_f$ con factor de amortiguamiento
 $k$. El péndulo oscila debido al efecto de la gravedad $F_{mg}$
La figura \ref{fig: simple pendulum} exhibe los componentes antes 
mencionados.


 \begin{figure}[ht]
    \centering
    \import{./img/}{pendulum_diagram.pdf_tex}
    \caption{Sistema de Péndulo simple.}
    \label{fig: simple pendulum}
\end{figure}

Para describir el comportamiento del sistema es necesario desarrollar
ecuaciones para detallar el movimiento y velocidad del cuerpo $m$.
Este trabajo se limitará a determinar las ecuaciones para un péndulo
cuyo movimiento está restringido a un plano sin la intervención de fuerzas
externas.

