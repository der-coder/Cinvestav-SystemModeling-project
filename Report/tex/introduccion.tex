\section{Introducción}

El péndulo simple (referido solamente como péndulo) es uno 
de los sistemas no lineares más estudiados por la 
comunidad científica y académica. La proliferación de su
uso como sistema base para evaluar estrategias de control ha
permitido desarrollar un conocimiento detallado del sistema.


El péndulo se compone de una mínima cantidad de elementos:
\begin{itemize}
 \item Un punto de rotación 
 \item Un objeto de masa $m$
 \item Un cuerpo rígido de longitud $l$ y masa despreciable que conecta el
 punto de rotación con el objeto de interés
\end{itemize}

La figura \ref{fig: simple pendulum} exhibe los componentes antes 
mencionados.


 \begin{figure}[ht]
    \centering
    \import{./img/}{pendulum_diagram.pdf_tex}
    \caption{Sistema de Péndulo simple.}
    \label{fig: simple pendulum}
\end{figure}

Para describir el comportamiento del sistema es necesario desarrollar
ecuaciones para detallar el movimiento y velocidad del cuerpo $m$.
Este trabajo se limitará a determinar las ecuaciones para un péndulo
cuyo movimiento está restringido a un plano sin la intervención de fuerzas
externas.

