\section{Introducción}

El péndulo simple ha sido uno de los mecanismos más estudiados 
por la comunidad científica a lo largo del tiempo, 
y sus aplicaciones han sido vastas. 
Galileo Galilei describió el comportamiento de
este mecanismo en 1602 \cite{drake2003galileo}, 
llegando a la conclusión de que el movimiento del mismo es 
\emph{isócrono}.
Quizás la aplicación más importante del péndulo simple ha sido
la invención del físico y matemático Christian Huygens: el reloj
de péndulo.
Inventado en 1656 y patentado en 1657 \cite{bennet2002huygenclock}, 
el reloj de péndulo demostró ser el instrumento más preciso para
la medición del tiempo hasta la construcción del reloj de
cuarzo en 1927 en Bell Laboratories \cite{morrison1948quartzcrystalclock}.\\

Es evidente que la comprensión detallada de este mecanismo
ha dado lugar a avances científicos e invenciones importantes 
a lo largo de la historia de la humanidad.
Este trabajo pretende mostrar un análisis de dicho mecanismo
empleando dos metodologías para su análisis.
El trabajo se enfoca en tres áreas de interés para el estudio
del péndulo simple: el modelo matemático del sistema, 
la simulación del modelo matemático empleando 
herramientas computacionales y una implementación
física del mecanismo.
