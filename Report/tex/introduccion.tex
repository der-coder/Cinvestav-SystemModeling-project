
El péndulo simple es quizá el sistema físico más empleado en 
academia para introducir conceptos de física y teoría de control.
Su estudio como sistema lineal y como sistema no lineal permite el
uso de múltiples técnicas de análisis para predecir su comportamiento
así como controlar el mismo. Para el caso de mínima complejidad, 
el sistema es modelado como un sistema de masa puntual 
operando en un rango de movimiento angular menor a 20 grados.
Estas consideraciones permiten linearizar el sistema al 
reemplazar funciones trascendentales que describen 
su movimiento por ecuaciones polinomiales.
