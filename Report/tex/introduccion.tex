
El péndulo simple puede considerarse como uno de los sistemas 
físicos más utilizados en  para introducir conceptos de física 
y teoría de control en el ámbito de educación. 
A pesar de su naturaleza como sistema no lineal, es posible tratarlo como
un sistema lineal al restringir su movimiento angular $\theta$ a un rango
no mayor a 20 grados medido desde el eje vertical.

Como se observa en la figura \ref{fig: simple pendulum}, 
el péndulo simple consta de un objeto de masa $m$ suspendido 
en el aire mediante un cuerpo rígido de longitud $l$. 
Este cuerpo rígido se asume de masa negligible para el estudio 
del sistema en este trabajo.

 \begin{figure}[ht]
    \centering
    \import{./img/}{pendulum_diagram.pdf_tex}
    \caption{Sistema de Péndulo simple.}
    \label{fig: simple pendulum}
\end{figure}

