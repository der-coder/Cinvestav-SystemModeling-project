\section{Marco Teórico}

\subsection{Péndulo simple}

Para su estudio, el péndulo simple se describe como una masa 
\emph{m} concentrada en un punto \cite{sastry2013nonlinear} 
que se encuentra suspendida mediante un elemento de masa 
despreciable y de longitud \emph{l} que la conecta a un 
punto de pivoteo.
Para el trabajo presentado, las fuerzas que actúan
sobre dicha masa se restringen a la fuerza de gravedad $F_{mg}$, 
la cual induce el movimiento del cuerpo, y una
fuerza de fricción viscosa $F_f$, la cual amortigua el sistema
y lo lleva al reposo después de un tiempo dado.\\

La representación del péndulo simple puede ser observada en la
figura \ref{fig: simple pendulum}. 
La posición angular $\theta$ de la masa es medida 
respecto a la vertical del sistema.
A su vez, la figura \ref{fig: pendulum forces} presenta las fuerzas que
actúan sobre la masa.

\begin{figure}[ht]
    \centering
    \import{./img/}{pendulum_diagram.pdf_tex}
    \caption{Sistema de Péndulo simple.}
    \label{fig: simple pendulum}
\end{figure}

 \begin{figure}[ht]
    \centering
    \import{./img/}{pendulum_forces.pdf_tex}
    \caption{Diagrama de fuerzas.}
    \label{fig: pendulum forces}
\end{figure}


\subsection{Leyes de movimiento de Newton}
% https://en.wikisource.org/wiki/The_Mathematical_Principles_of_Natural_Philosophy_(1729)

Sir Isaac Newton establece en su obra \emph{Principia Mathematica} 
las tres leyes que fungen como el cimiento de la mecánica clásica 
\cite{newton1803mathematical, díaz20183d}. 
A continuación se presentan las leyes mencionadas.

\begin{enumerate}
 \item Un cuerpo mantiene su estatus quo, 
 respecto a un marco referencial inercial, salvo
 que una fuerza externa actúe sobre éste.\\
 
 La preservación del estatus quo de un cuerpo establece
 que la tasa de cambio de la velocidad $\bold{v}$ 
 del cuerpo es cero.
 
 \begin{equation}
  \sum \bold{F} = 0 \iff \dfrac{d \bold{v}}{dt} = 0
  \label{eq: 1st law of motion}
 \end{equation}

 \item La tasa de cambio de la cantidad de movimiento 
 (momentum) $\bold{p}$ de un cuerpo
 es directamente proporcional a la fuerza aplicada.
 \begin{equation}
  \bold{F} = \dfrac{d \bold{p}}{dt}
  \label{eq: 2nd law of motion}
 \end{equation}
 
 La cantidad de movimiento de un cuerpo se define como el 
 producto de la masa del cuerpo y el vector de velocidad 
 del mismo ($\bold{p} = m \bold{v}$). 
 Con esta definición es posible expresar 
 \eqref{eq: 2nd law of motion} de la siguiente manera.
 
 \begin{equation}
  \begin{split}
   \bold{F} &= \dfrac{d (m\bold{v})}{dt}\\
   &= m \bold{a}
  \end{split}
  \label{eq: conservation of linear momentum}
 \end{equation}


 \item El efecto mutuo de dos cuerpos que actúan 
 uno sobre el otro es siempre igual y en direcciones contrarias.
 \begin{equation}
  \bold{F}_{a/b} = -\bold{F}_{b/a}
  \label{eq: 3rd law of motion}
 \end{equation}

\end{enumerate}


Las leyes de Newton pueden ser aplicadas tanto a 
movimientos lineales como movimientos rotacionales.
Para el caso de la segunda ley de Newton, 
es posible extender el concepto de tasa de cambio
de la cantidad de movimiento del cuerpo 
a la tasa de cambio del momento angular $\bold{L}$
del cuerpo.
Para esta nueva expresión, el momento angular se relaciona
con el efecto que tiene el torque $\boldsymbol{\tau}$
que genera la fuerza sobre el cuerpo en cuestión.

\begin{equation}
  \sum \boldsymbol{\tau} = \dfrac{d \bold{L}}{dt}
  \label{eq: conservation of angular momentum}
\end{equation}

Considerando que el momento angular para una partícula
se define como el producto del momento de inercia 
y la velocidad angular
($\bold{L} = I \boldsymbol{\omega}$), es posible 
expresar a \eqref{eq: conservation of angular momentum}
de la siguiente manera:

\begin{equation}
 \begin{split}
  \sum \boldsymbol{\tau} &= \dfrac{d (I \boldsymbol{\omega})}{dt} \\
  &= I \boldsymbol{\alpha}
 \end{split}
 \label{eq: angular momentum and torque}
\end{equation}

\subsection{Energía}

Para una partícula solamente se consideran dos tipos de 
energía \cite{susskind2014theoretical}: la energía cinética,
relacionada con el movimiento de la partícula, y la 
energía potencial, que es una función que depende de la 
posición de la partícula en el espacio.

\begin{itemize}
 \item Energía cinética.\\
 De acuerdo a \cite{díaz20183d}, 
 la energía cinética de una partícula se expresa de 
 la siguiente manera: 
 \begin{equation}
  T = \dfrac{1}{2} m ||\bold{v}||^2
  \label{eq: kinetic energy}
 \end{equation}

 
 \item Energía potencial.\\
 La energía potencial de la partícula es una función 
 que depende de la posición de partícula, los detalles de esta 
 ecuación se presentarán en el desarrollo matemático
 del modelo.
 \begin{equation}
    V = V(x)
  \label{eq: potential energy}
 \end{equation}

\end{itemize}

\subsection{Mecánica Lagrangiana}

La mecánica Lagrangiana es un replanteamiento de la 
mecánica clásica Newtoniana, haciendo uso
de \emph{coordenadas generalizadas}
$\bold{q} = 
\begin{pmatrix}
q_1 & q_2 &\cdots & q_i
\end{pmatrix}^T
$ para describir un sistema.
Para sistemas como el péndulo simple, su uso simplifica el
desarrollo matemático del mismo.\\

Se introduce el concepto del \emph{Lagrangiano} $\mathcal{L}$,
que es una ecuación que describe la dinámica del sistema en 
función de la energía.

\begin{equation}
 \mathcal{L} = T - V(x)
 \label{eq: lagrangian}
\end{equation}

Se plantea la ecuación de Euler-Lagrange, 
que para el cálculo de variaciones permite determinar
puntos estacionarios para la ecuación funcional del sistema.
Esto permite minimizar la acción $\mathcal{A}$ del sistema.

\begin{equation}
 \dfrac{d}{dt} \dfrac{\partial \mathcal{L}}{\partial \dot{q}} - 
 \dfrac{\partial \mathcal{L}}{\partial q} = 0
 \label{eq: euler lagrange equation}
\end{equation}




%  Euler-Lagrange equation
