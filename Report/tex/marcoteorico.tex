\section{Marco Teórico}

\subsection{Leyes de movimiento de Newton}
% De acuerdo a la traducción realizada por Andrew Motte
% http://gravitee.tripod.com/
% https://en.wikisource.org/wiki/The_Mathematical_Principles_of_Natural_Philosophy_(1729)

Sir Isaac Newton establece en su obra \emph{Principia Mathematica} las tres leyes que fungen como la fundación de la mecánica clásica.

\begin{enumerate}
 \item Un cuerpo mantiene su estatus quo, 
 respecto a un marco referencial inercial, salvo
 que una fuerza externa actué sobre éste. (Ley de inercia)
 \begin{equation}
  \sum \bold{F} = 0 \iff \dfrac{d \bold{v}}{dt} = 0
 \end{equation}

 \item La tasa de cambio de la cantidad de movimiento (momentum) de un cuerpo
 es directamente proporcional a la fuerza aplicada. (Conservación del momentum)
 \begin{equation}
  \bold{F} = \dfrac{d \bold{p}}{dt}
  \label{eq: 2nd law of motion}
 \end{equation}

 \item El efecto mútuo de dos cuerpos que actúan 
 uno sobre el otro es siempre igual y en direcciones contrarias.
 \begin{equation}
  \bold{F}_{a/b} = -\bold{F}_{b/a}
 \end{equation}

\end{enumerate}

Dado que la cantidad de movimiento es descrita en función de la masa ($m$)
del objeto y su velocidad lineal ($\bold{v}$), 
es posible expresar \eqref{eq: 2nd law of motion} de la siguiente manera
\begin{equation}
 \begin{split}
  \sum \bold{F} & = \dfrac{d \bold{p}}{dt}\\
  \bold{p} & = m \bold{v}\\
  \sum \bold{F} & = \dfrac{d m\bold{v}}{dt}\\
%   \sum \bold{F} & = m\dfrac{d\bold{v}}{dt}\\
  \sum \bold{F} & = m\dfrac{d\bold{a}}{dt}
 \end{split}
 \label{eq: conservation of linear momentum}
\end{equation}

De manera similar, la conservación de movimiento \emph{angular} ($L$), 
que relaciona el momento de inercia ($I$) de un cuerpo con su 
velocidad angular ($\omega$),
se expresa en función del torque que se aplica en un cuerpo 
respecto a un marco de referencia.

\begin{equation}
 \begin{split}
  \sum \boldsymbol{\tau} &= \dfrac{d \bold{L}}{dt} \\
  \bold{L} & = I \boldsymbol{\omega}\\
  \sum \boldsymbol{\tau} &= \dfrac{d I \boldsymbol{\omega}}{dt} \\
  \sum \boldsymbol{\tau} &= I \dfrac{d \boldsymbol{\alpha}}{dt} 
 \end{split}
 \label{eq: conservation of angular momentum}
\end{equation}

% 2nd Law of motion for rotational systems

\subsection{Conservación de energía}

%  Euler-Lagrange equation
