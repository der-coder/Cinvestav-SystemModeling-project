
La posición del objeto $m$ respecto a un marco de referencia no inercial
ubicado en el punto de rotación del péndulo puede ser descrita como un par
de coordenadas $\{x, y\}$ para un plano cartesiano. Estas coordenadas pueden
ser expresadas también como funciones que dependen de la posición angular 
$\theta = \theta(t)$ del objeto con respecto al eje vertical del plano cartesiano y el
tiempo $t$.

Existen dos metodologías para obtener las ecuaciones para las coordenadas 
$\{x, y\}$ del sistema:
\begin{itemize}
 \item Balance de fuerzas del sistema empleando las leyes de 
 movimiento de Newton.
 \item Conservación de energía mediante la ecuación de Euler-Lagrange
\end{itemize}



%  --------------------------
%  Newton's Laws of Motion
% ---------------------------

\subsection{Leyes de movimiento de Newton}

Empleando las leyes de movimiento de Newton, 
se establece que la aceleración $\textbf{a}$ que el objeto
experimenta está determinada por el el
efecto combinado de las fuerzas que actúan sobre el mismo.
Como se observa en la figura \ref{fig: pendulum forces}, 
para el caso de un péndulo simple sin fuerzas externas 
hay dos vectores de fuerza presentes en el sistema:
la fuerza de gravedad $\textbf{F}_{mg} = mg$ que actúa paralelo al eje vertical 
y la fuerza de fricción $\textbf{F}_f = kl\dot{\theta}$ que se opone al movimiento del objeto. 

 \begin{figure}[ht]
    \centering
    \import{./img/}{pendulum_forces.pdf_tex}
    \caption{Diagrama de fuerzas.}
    \label{fig: pendulum forces}
\end{figure}

De acuerdo al diagrama y empleando las leyes de movimiento de Newton, 
el efecto combinado de las fuerzas en el objeto se expresan de la 
siguiente manera:


\begin{equation}
 \begin{split}
  \sum \textbf{F} & = m \textbf{a}\\
  \sum \bold{F} & = - \bold{F}_{mg} - \bold{F}_f
 \end{split}
 \label{eq: }
\end{equation}

Debido al movimiento circular del sistema, la fuerza de gravedad
se descompone en dos vectores ortogonales que actúan cada uno sobre un eje 
diferente. Esta descomposición del vector permite despreciar
el efecto de uno de ellos para el estudio del sistema.

\begin{equation*}
 \begin{split}
%   \sum F_y &= -mg \cos{\theta} \\
  \sum F &= -mg \sin{\theta} - k l \dot{\theta}\\
  m l \ddot{\theta} &= -mg \sin{\theta} - k l \dot{\theta}\\
 \end{split}
\end{equation*}

Se resuelve la ecuación anterior para $\ddot{\theta}$.

\begin{equation}
 \ddot{\theta} = - \dfrac{g}{l} \sin{\theta} - \dfrac{k}{m} \dot{\theta}
 \label{eq: newton equation}
\end{equation}


\subsection{Conservación de energía | Euler-Lagrange}

El método de conservación de energía del sistema se sustenta
en la ecuación de Euler-Lagrange, la cual describe que para un
sistema dado la energía total del sistema permanece constante.

\begin{equation}
 \dfrac{d}{dt} \dfrac{\partial L}{\partial \dot{q}} - 
 \dfrac{\partial L}{\partial q} = 0
 \label{eq: euler lagrange equation}
\end{equation}

El Lagrangiano $L$ del sistema relaciona la energía cinética $T$ 
del péndulo con el potencial de energía $V$ en el sistema.

\begin{equation*}
\begin{split}
 T &= \dfrac{1}{2} m v^2\\
 &= \dfrac{1}{2} m (l\dot{\theta})^2\\
 &= \dfrac{1}{2} m l^2 \dot{\theta}^2\\
\end{split}
\label{eq: kinetic energy}
\end{equation*}

\begin{equation*}
 V = m g l \left( 1 - \cos{\theta} \right)
 \label{eq: potential energy}
\end{equation*}

Recordando la ecuación para el Lagrangiano $L = T - V$ se obtienen las 
derivadas parciales del mismo, considerando que $q = \theta$.

\begin{equation}
 \begin{split}
  L & = \dfrac{1}{2}m l^2 \dot{\theta}^2 - m g l (1 - \cos{\theta})\\
  \dfrac{\partial L}{\partial \theta} &= - m g l \sin{\theta} \\
  \dfrac{\partial L}{\partial \dot{\theta}} &= m l^2 \dot{\theta}\\
  \dfrac{d}{dt}\dfrac{\partial L}{\partial \dot{\theta}} &=ml^2 \ddot{\theta}\\
 \end{split}
 \label{eq: partial derivatives}
\end{equation}

Evaluando las derivadas parciales de \eqref{eq: partial derivatives} en 
la ecuación \eqref{eq: euler lagrange equation} se obtiene

\begin{equation}
 \begin{split}
    ml^2 \ddot{\theta} + m g l \sin{\theta} &= 0\\
    \ddot{\theta} &= -\dfrac{mgl \sin{\theta}}{ml^2}\\
    \ddot{\theta} &= - \dfrac{g}{l} \sin{\theta}
 \end{split}
\end{equation}

Cabe resaltar que para el método de conservación de 
energía las fuerzas no conservativas no 
fueron consideradas en el desarrollo matemático del modelo.
Por ello, si se desprecia el efecto de la fricción en 
\eqref{eq: newton equation} se observa que la ecuación
de $\ddot{\theta}$ es la misma para ambos métodos.

\subsection{Representación de espacio de estados}

