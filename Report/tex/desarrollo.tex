\section{Modelo Matemático}


La posición del objeto $m$ respecto a un marco de referencia no inercial
ubicado en el punto de rotación del péndulo puede ser descrita como un par
de coordenadas $\{x, y\}$ para un plano cartesiano. Estas coordenadas pueden
ser expresadas también como funciones que dependen de la posición angular 
$\theta$ del objeto con respecto al eje vertical del plano cartesiano y el
tiempo $t$.

Existen dos metodologías para obtener las ecuaciones para las coordenadas 
$\{x, y\}$ del sistema:
\begin{itemize}
 \item Balance de fuerzas del sistema empleando las leyes de 
 movimiento de Newton.
 \item Conservación de energía mediante la ecuación de Euler-Lagrange
\end{itemize}



%  --------------------------
%  Newton's Laws of Motion
% ---------------------------

\subsection{Leyes de movimiento de Newton}


% El péndulo simple es un sistema el cual se basa en una 
% particula de masa $m$ sostenido de un punto fijo por medio 
% de una barra o hilo de masa despreciable y sin extenderse 
% mas de su distancia $l$. (Insertar figura del péndulo)\\
% 
% Para encontrar el movimiento de un péndulo se utilizaron los 
% métodos de fuerza de Newton y el método de energías de 
% Lagrange. Por medio del método de fuerzas de Newton, se 
% desarrolla de la siguiente manera. \begin{large} 
% \begin{gather*} \sum F = ma = -F_{mg} - F_f \\ \bigskip 
% ml\ddot{\theta} = -mg\sin(\theta) - kl\dot{\theta} \\ 
% \bigskip \ddot{\theta} = -\dfrac{g}{l}\sin(\theta) - 
% \dfrac{k}{m}\dot{\theta} \\ \end{gather*} \end{large} 
% \begin{flushright} \begin{small} m = masa del péndulo\\ l = 
% largo del péndulo\\ k = constante fricción\\ \end{small} 
% \end{flushright}
% 
% El modelo en base al método de Newton se basa en conocer las 
% fuerzas actuando, las fuerzas principales que actúan sobre 
% el péndulo es la fuerza ocasionada por el peso de la masa 
% ($F_{mg}$) y la fuerza de al fricción que se opone al 
% movimiento del péndulo ($F_f$).\\

Empleando las leyes de movimiento de Newton, 
se establece que la aceleración $\textbf{a}$ que el objeto
experimenta está determinada por el el
efecto combinado de las fuerzas que actúan sobre el mismo.
Como se observa en la figura \ref{fig: pendulum forces}, 
para el caso de un péndulo simple sin fuerzas externas 
hay dos vectores de fuerza presentes en el sistema:
la fuerza de gravedad $\textbf{F}_{mg} = mg$ que actúa paralelo al eje vertical 
y la fuerza de fricción $\textbf{F}_f = kl\dot{\theta}$ que se opone al movimiento del objeto. 

 \begin{figure}[ht]
    \centering
    \import{./img/}{pendulum_forces.pdf_tex}
    \caption{Diagrama de fuerzas.}
    \label{fig: pendulum forces}
\end{figure}

De acuerdo al diagrama y empleando las leyes de movimiento de Newton, 
el efecto combinado de las fuerzas en el objeto se expresan de la 
siguiente manera:


\begin{equation}
 \begin{split}
  \sum \textbf{F} & = m \textbf{a}\\
  \sum \bold{F} & = - \bold{F}_{mg} - \bold{F}_f
 \end{split}
 \label{eq: }
\end{equation}

Debido al movimiento circular del sistema, la fuerza de gravedad
se descompone en dos vectores ortogonales que actúan cada uno sobre un eje 
diferente. Esta descomposición del vector permite despreciar
el efecto de uno de ellos para el estudio del sistema.

\begin{equation*}
 \begin{split}
%   \sum F_y &= -mg \cos{\theta} \\
  \sum F &= -mg \sin{\theta} - k l \dot{\theta}\\
  m l \ddot{\theta} &= -mg \sin{\theta} - k l \dot{\theta}\\
 \end{split}
\end{equation*}

Se resuelve la ecuación anterior para $\ddot{\theta}$.

\begin{equation}
 \ddot{\theta} = - \dfrac{g}{l} \sin{\theta} - \dfrac{k}{m} \dot{\theta}
\end{equation}


\subsection{Conservación de energía | Euler-Lagrange}

El método de conservación de energía del sistema se sustenta
en la ecuación de Euler-Lagrange, la cual describe que para un
sistema dado la energía total del sistema permanece constante.

\begin{equation}
 \dfrac{d}{dt} \dfrac{\partial L}{\partial \dot{q}} - 
 \dfrac{\partial L}{\partial q} = 0
 \label{eq: euler lagrange equation}
\end{equation}


Por este método se necesitan desarrollar las ecuaciones de energía del péndulo. Y para desarrollar las ecuaciones de posición asignamos el marco de referencia del péndulo y se obtiene:\\
\begin{equation}
\left(\begin{matrix}
x\\
y
\end{matrix}\right) = 
\begin{bmatrix}
l\sen(\theta)\\
l(1 - \cos(\theta)
\end{bmatrix}
\end{equation}


El Lagrangiano $L$ del sistema relaciona la energía cinética $T$ 
del péndulo con el potencial de energía $V$ en el sistema.

\begin{equation*}
\begin{split}
 T &= \dfrac{1}{2} m v^2\\
 &= \dfrac{1}{2} m (l\dot{\theta})^2\\
 &= \dfrac{1}{2} m l^2 \dot{\theta}^2\\
\end{split}
\label{eq: kinetic energy}
\end{equation*}

\begin{equation*}
 V = m g l \left( 1 - \cos{\theta} \right)
 \label{eq: potential energy}
\end{equation*}

Recordando la ecuación para el Lagrangiano $L = T - V$ se obtienen las 
derivadas parciales del mismo, considerando que $q = \theta$.

\begin{equation}
 \begin{split}
  L & = \dfrac{1}{2}m l^2 \dot{\theta}^2 - m g l (1 - \cos{\theta})\\
  \dfrac{\partial L}{\partial \theta} &= - m g l \sin{\theta} \\
  \dfrac{\partial L}{\partial \dot{\theta}} &=\\
  \dfrac{d}{dt}\dfrac{\partial L}{\partial \dot{\theta}} &=\\
 \end{split}
\end{equation}

\begin{flushleft}
{\large Energía Cinética}
\end{flushleft}
\begin{equation} \label{T_equ}
T = \frac{1}{2}mv^2 = \frac{1}{2}m(\dot{x}^2 + \dot{y}^2) 
\end{equation}
Derivando las ecuaciones de posición del péndulo obtenemos las ecuaciones de velocidad:
\begin{equation} \label{Vel_p}
\left(\begin{matrix}
\dot{x}\\
\dot{y}
\end{matrix}\right) = 
\begin{bmatrix}
l\dot{\theta}\cos(\theta)\\
l\dot{\theta}\sen(\theta)
\end{bmatrix}
\end{equation}
Sustituyendo la ecuación \ref{Vel_p} en \ref{T_equ} y desarrollando se obtiene:
\begin{equation}
T = \frac{1}{2}ml^2\dot{\theta}^2
\end{equation}
\begin{flushleft}
{\large Energía Potencial}
\end{flushleft}
La energía potencial se plantea multiplicando la posición en el \emph{eje y} del péndulo por la masa y gravedad. Se plantea de la siguiente manera:
\begin{equation} \label{V_equ}
V = mgl(1-\cos \theta)
\end{equation}
Con estas ecuaciones se puede definir el Lagrangiano el cual es el que va a ser diferenciado por medio de la ecuación \ref{L_equ}. El Lagrangiano se define como:

\begin{large}
\begin{align*}
L = T - V = \frac{1}{2}m(l\theta)^2 \\
= \frac{1}{2}ml^2\dot{\theta}^2 - mgl(1-\cos \theta)
\end{align*}
\end{large}

\begin{large}
\begin{equation*}
\dfrac{d}{dt} \dfrac{\partial L}{\partial \dot{\theta}} = \dfrac{d}{dt} ml^2\dot{\theta}
\end{equation*}
\end{large}
\begin{large}
\begin{equation} \label{dLv_equ}
\dfrac{d}{dt} ml^2\dot{\theta} = ml^2\ddot{\theta}
\end{equation}
\end{large}
\begin{large}
\begin{equation} \label{dLp_equ}
\dfrac{\partial L}{\partial\theta} = -gl\sin(\theta)
\end{equation}
\end{large}
Al unir la ecuación \ref{dLv_equ} menos la ecuación \ref{dLp_equ}, en base a la diferenciación del Lagrangiano (ecuación \ref{L_equ}), se obtiene la ecuación de movimiento del sistema.
\begin{large}
\begin{equation}
ml^2\ddot{\theta} + gl\sin(\theta) = 0
\end{equation}
\end{large}
