
\subsection{Modelo matemático}

Para describir el movimiento de un sistema de péndulo simple 
restringido al plano bidimensional es necesario expresar 
la posición y velocidad de la masa $m$ en función del movimiento
angular $\theta$ y las fuerzas que actúan sobre el sistema.
Se utilizaron dos metodologías para obtener las ecuaiones
de movimiento del sistema: 
\begin{itemize}
 \item Suma de fuerzas del sistema empleando las leyes de movimiento de Newton.
 \item Conservación de energía mediante la ecuación de Euler-Lagrange.
\end{itemize}

%  --------------------------
%  Newton's Laws of Motion
% ---------------------------

Partiendo de las leyes de movimiento de Newton, se establece que la aceleración
del objeto de masa $m$ puede ser descrita en función de las fuerzas 
presentes en el sistema. Para el caso de un péndulo simple se tienen dos fuerzas:
la fuerza de gravedad $F_g$ que actúa sobre todos los cuerpos 
y la fuerza de fricción $F_f$ que se opone al movimiento del objeto.

\begin{large}
\begin{gather*}
\sum F = ma = -F_{mg} - F_f \\ \bigskip
ml\ddot{\theta} = -mg\sin(\theta) - kl\dot{\theta} \\ \bigskip
\ddot{\theta} = -\dfrac{g}{l}\sin(\theta) - \dfrac{k}{m}\dot{\theta} \\
\end{gather*} 
\end{large}
\begin{flushright}
\begin{small}
m = masa del péndulo\\
l = largo del péndulo\\
k = constante fricción\\
\end{small}
\end{flushright}

El modelo en base al método de Newton se basa en conocer las fuerzas actuando, las fuerzas principales que actúan sobre el péndulo es la fuerza ocasionada por el peso de la masa ($F_{mg}$) y la fuerza de al fricción que se opone al movimiento del péndulo ($F_f$).\\

El segundo método utilizado para encontrar las ecuaciones de movimiento fue el de energías de Lagrange.
\begin{large}
\begin{equation} \label{L_equ}
\dfrac{d}{dt} \dfrac{\partial L}{\partial \dot{\theta}} - \dfrac{\partial L}{\partial\theta} = 0
\end{equation}
\end{large}
Por este método se necesitan desarrollar las ecuaciones de energía del péndulo.\\
\begin{flushleft}
{\large Energía Cinética}
\end{flushleft}
\begin{equation} \label{T_equ}
T = \frac{1}{2}mv^2 = \frac{1}{2}m(l\theta)^2 = \frac{1}{2}ml^2\theta^2 
\end{equation}
\begin{flushleft}
{\large Energía Potencial}
\end{flushleft}
\begin{equation} \label{V_equ}
V = mgl(1-\cos \theta)
\end{equation}
Con estas ecuaciones se puede definir el Lagrangiano el cual es el que va a ser diferenciado por medio de la ecuación \ref{L_equ}. El Lagrangiano se define como:

\begin{large}
\begin{align*}
L = T - V = \frac{1}{2}m(l\theta)^2 \\
= \frac{1}{2}ml^2\theta^2 - mgl(1-\cos \theta)
\end{align*}
\end{large}

\begin{large}
\begin{equation*}
\dfrac{d}{dt} \dfrac{\partial L}{\partial \dot{\theta}} = \dfrac{d}{dt} ml^2\dot{\theta}
\end{equation*}
\end{large}
\begin{large}
\begin{equation} \label{dLv_equ}
\dfrac{d}{dt} ml^2\dot{\theta} = ml^2\ddot{\theta}
\end{equation}
\end{large}
\begin{large}
\begin{equation} \label{dLp_equ}
\dfrac{\partial L}{\partial\theta} = -gl\sin(\theta)
\end{equation}
\end{large}
Al unir la ecuación \ref{dLv_equ} menos la ecuación \ref{dLp_equ}, en base a la diferenciación del Lagrangiano (ecuación \ref{L_equ}), se obtiene la ecuación de movimiento del sistema.
\begin{large}
\begin{equation}
ml^2\ddot{\theta} + gl\sin(\theta) = 0
\end{equation}
\end{large}
