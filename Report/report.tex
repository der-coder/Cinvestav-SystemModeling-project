% \documentclass[conference, onecolumn]{IEEEtran}

\documentclass[letterpaper, 12pt]{article}

\usepackage{hyperref}

\usepackage{amsmath,amssymb,amsfonts}
\usepackage{algorithmic}
\usepackage{graphicx}
\usepackage{textcomp}
\usepackage{xcolor}
\usepackage[spanish,mexico]{babel}

\usepackage{biblatex}
\addbibresource{bibliografia.bib}

\graphicspath{{img/}}

\usepackage{float}
\usepackage{import}
\usepackage{xifthen}
\usepackage{pdfpages}
\usepackage{transparent}
\usepackage[center]{caption}
\usepackage{bm}

    \usepackage{listings}
    \lstset{ 
    	language=Matlab,                		% choose the language of the code
    %	basicstyle=10pt,       				% the size of the fonts that are used for the code
    	numbers=left,                  			% where to put the line-numbers
    	numberstyle=\footnotesize,      		% the size of the fonts that are used for the line-numbers
    	stepnumber=1,                   			% the step between two line-numbers. If it's 1 each line will be numbered
    	numbersep=5pt,                  		% how far the line-numbers are from the code
    %	backgroundcolor=\color{white},  	% choose the background color. You must add \usepackage{color}
    	showspaces=false,               		% show spaces adding particular underscores
    	showstringspaces=false,         		% underline spaces within strings
    	showtabs=false,                 			% show tabs within strings adding particular underscores
    %	frame=single,	                			% adds a frame around the code
    %	tabsize=2,                				% sets default tabsize to 2 spaces
    %	captionpos=b,                   			% sets the caption-position to bottom
    	breaklines=true,                			% sets automatic line breaking
    	breakatwhitespace=false,        		% sets if automatic breaks should only happen at whitespace
    	escapeinside={\%*}{*)}          		% if you want to add a comment within your code
    }


\begin{document}

\title{
Péndulo Simple\\
Análisis, simulación y construcción\\
}


\author{
    Enrique Benavides Téllez, Isaac Ayala Lozano,\\ 
    Sandy Natalie Campos Martínez,\\
    Luis Gerardo Almanza Granados\\
    y Yair Casas Flores
}

\date{}

\maketitle

\begin{abstract}

% El sistema del péndulo simple es uno de los sistemas más estudiados
% en teoría de control. 
% Su diseño y contrucción de baja dficultad hacen de este sistema uno de
% los más accesibles para modelar y contrastar con una implementación
% física.
% El documento presente muestra el estudio del sistema sin linearización.
% El estudio comprende la obtención de las ecuaciones de movimiento,
% la simulación del mismo en MATLAB y la comparación del modelo con
% un sistema físico.


Se presenta el desarrollo de un modelo matemático
para el péndulo simple mediante tres formulaciones de mecánica: 
mecánica Newtoniana, mecánica Lagrangiana y mecánica Hamiltoniana.
Se incluyen los resultados de los simuladores desarrollados 
en MATLAB.
Se hace una comparación de los modelos matemáticos 
con una implementación física del péndulo simple empleando el 
kit de LEGO Mindstorm.

\end{abstract}

% % \begin{IEEEkeywords}
% \end{IEEEkeywords}


\section{Introducción}

El péndulo simple ha sido uno de los mecanismos más estudiados 
por la comunidad científica a lo largo del tiempo, 
y sus aplicaciones han sido vastas. 
Galileo Galilei describió el comportamiento de
este mecanismo en 1602 \cite{drake2003galileo}, 
llegando a la conclusión de que el movimiento del mismo es 
\emph{isócrono}.
Quizás la aplicación más importante del péndulo simple ha sido
la invención del físico y matemático Christian Huygens: el reloj
de péndulo.
Inventado en 1656 y patentado en 1657 \cite{bennet2002huygenclock}, 
el reloj de péndulo demostró ser el instrumento más preciso para
la medición del tiempo hasta la construcción del reloj de
cuarzo en 1927 en Bell Laboratories \cite{morrison1948quartzcrystalclock}.\\

Es evidente que la comprensión detallada de este mecanismo
ha dado lugar a avances científicos e invenciones importantes 
a lo largo de la historia de la humanidad.
Este trabajo pretende mostrar un análisis de dicho mecanismo
empleando dos metodologías para su análisis.
El trabajo se enfoca en tres áreas de interés para el estudio
del péndulo simple: el modelo matemático del sistema, 
la simulación del modelo matemático empleando 
herramientas computacionales y una implementación
física del mecanismo.

\section{Marco Teórico}

\subsection{Péndulo simple}

Para su estudio, el péndulo simple se describe como una masa 
\emph{m} concentrada en un punto \cite{sastry2013nonlinear} 
que se encuentra suspendida mediante un elemento de masa 
despreciable y de longitud \emph{l} que la conecta a un 
punto de pivoteo.
Para el trabajo presentado, las fuerzas que actúan
sobre dicha masa se restringen a la fuerza de gravedad $F_{mg}$, 
la cual induce el movimiento del cuerpo, y una
fuerza de fricción viscosa $F_f$, la cual amortigua el sistema
y lo lleva al reposo después de un tiempo dado.\\

La representación del péndulo simple puede ser observada en la
figura \ref{fig: simple pendulum}. 
La posición angular $\theta$ de la masa es medida 
respecto a la vertical del sistema.
A su vez, la figura \ref{fig: pendulum forces} presenta las fuerzas que
actúan sobre la masa.

\begin{figure}[ht]
    \centering
    \import{./img/}{pendulum_diagram.pdf_tex}
    \caption{Sistema de Péndulo simple.}
    \label{fig: simple pendulum}
\end{figure}

 \begin{figure}[ht]
    \centering
    \import{./img/}{pendulum_forces.pdf_tex}
    \caption{Diagrama de fuerzas.}
    \label{fig: pendulum forces}
\end{figure}


\subsection{Leyes de movimiento de Newton}
% https://en.wikisource.org/wiki/The_Mathematical_Principles_of_Natural_Philosophy_(1729)

Sir Isaac Newton establece en su obra \emph{Principia Mathematica} 
las tres leyes que fungen como el cimiento de la mecánica clásica 
\cite{newton1803mathematical, díaz20183d}. 
A continuación se presentan las leyes mencionadas.

\begin{enumerate}
 \item Un cuerpo mantiene su estatus quo, 
 respecto a un marco referencial inercial, salvo
 que una fuerza externa actúe sobre éste.\\
 
 La preservación del estatus quo de un cuerpo establece
 que la tasa de cambio de la velocidad $\bold{v}$ 
 del cuerpo es cero.
 
 \begin{equation}
  \sum \bold{F} = 0 \iff \dfrac{d \bold{v}}{dt} = 0
  \label{eq: 1st law of motion}
 \end{equation}

 \item La tasa de cambio de la cantidad de movimiento 
 (momentum) $\bold{p}$ de un cuerpo
 es directamente proporcional a la fuerza aplicada.
 \begin{equation}
  \bold{F} = \dfrac{d \bold{p}}{dt}
  \label{eq: 2nd law of motion}
 \end{equation}
 
 La cantidad de movimiento de un cuerpo se define como el 
 producto de la masa del cuerpo y el vector de velocidad 
 del mismo ($\bold{p} = m \bold{v}$). 
 Con esta definición es posible expresar 
 \eqref{eq: 2nd law of motion} de la siguiente manera.
 
 \begin{equation}
  \begin{split}
   \bold{F} &= \dfrac{d (m\bold{v})}{dt}\\
   &= m \bold{a}
  \end{split}
  \label{eq: conservation of linear momentum}
 \end{equation}


 \item El efecto mutuo de dos cuerpos que actúan 
 uno sobre el otro es siempre igual y en direcciones contrarias.
 \begin{equation}
  \bold{F}_{a/b} = -\bold{F}_{b/a}
  \label{eq: 3rd law of motion}
 \end{equation}

\end{enumerate}


Las leyes de Newton pueden ser aplicadas tanto a 
movimientos lineales como movimientos rotacionales.
Para el caso de la segunda ley de Newton, 
es posible extender el concepto de tasa de cambio
de la cantidad de movimiento del cuerpo 
a la tasa de cambio del momento angular $\bold{L}$
del cuerpo.
Para esta nueva expresión, el momento angular se relaciona
con el efecto que tiene el torque $\boldsymbol{\tau}$
que genera la fuerza sobre el cuerpo en cuestión.

\begin{equation}
  \sum \boldsymbol{\tau} = \dfrac{d \bold{L}}{dt}
  \label{eq: conservation of angular momentum}
\end{equation}

Considerando que el momento angular para una partícula
se define como el producto del momento de inercia 
y la velocidad angular
($\bold{L} = I \boldsymbol{\omega}$), es posible 
expresar a \eqref{eq: conservation of angular momentum}
de la siguiente manera:

\begin{equation}
 \begin{split}
  \sum \boldsymbol{\tau} &= \dfrac{d (I \boldsymbol{\omega})}{dt} \\
  &= I \dfrac{d \boldsymbol \omega}{dt}\\
  &= I \boldsymbol{\alpha}
 \end{split}
 \label{eq: angular momentum and torque}
\end{equation}

Donde $\boldsymbol \alpha$ es la aceleración angular de la partícula.

\subsection{Energía}

Para una partícula solamente se consideran dos tipos de 
energía \cite{susskind2014theoretical}: la energía cinética,
relacionada con el movimiento de la partícula, y la 
energía potencial, que es una función que depende de la 
posición de la partícula en el espacio.

 \textbf{Energía cinética.}
 De acuerdo a \cite{díaz20183d}, 
 la energía cinética de una partícula se expresa como una 
 función que depende de la masa \emph m y 
 el vector de velocidad $\bold v$ de
 la siguiente manera: 
 \begin{equation}
  T = \dfrac{1}{2} m ||\bold v ||^2
  \label{eq: kinetic energy}
 \end{equation}

 En donde $||\bold v ||$ se define como $||\bold v|| = \sqrt{\bold v \cdot \bold v}$.\\
 
 \textbf{Energía potencial.}  
 La energía potencial de la partícula es una función 
 que depende de la posición de partícula, los detalles de esta 
 ecuación se presentarán en el desarrollo matemático
 del modelo.
 \begin{equation}
    V = V(x)
  \label{eq: potential energy}
 \end{equation}


\subsection{Mecánica Lagrangiana}

La mecánica Lagrangiana es un replanteamiento de la 
mecánica clásica Newtoniana, haciendo uso
de \emph{coordenadas generalizadas}
$\bold{q} = 
\begin{pmatrix}
q_1 & q_2 &\cdots & q_i
\end{pmatrix}^T
$ para describir un sistema.
Para sistemas como el péndulo simple, su uso simplifica el
desarrollo matemático del mismo.\\

Se introduce el concepto del \emph{Lagrangiano} $\mathcal{L}$,
que es una ecuación que describe la dinámica del sistema en 
función de la energía cinética $T$ y la energía potencial $V(x)$.

\begin{equation}
 \mathcal{L} = T - V(x)
 \label{eq: lagrangian}
\end{equation}

Se plantea la ecuación de Euler-Lagrange, 
que para el cálculo de variaciones permite determinar
puntos estacionarios para la ecuación funcional del sistema.
Esto permite minimizar la acción $\mathcal{A}$ del sistema.

\begin{equation}
 \dfrac{d}{dt} \dfrac{\partial \mathcal{L}}{\partial \dot{q}} - 
 \dfrac{\partial \mathcal{L}}{\partial q} = 0
 \label{eq: euler lagrange equation}
\end{equation}



\subsection{Mecánica Hamiltoniana}
% https://www.macs.hw.ac.uk/~simonm/mechanics.pdf
% http://www.people.fas.harvard.edu/~djmorin/chap15.pdf
% https://cds.cern.ch/record/399399/files/p1.pdf
% http://www.damtp.cam.ac.uk/user/tong/dynamics.html
% https://www.maths.tcd.ie/pub/HistMath/People/Hamilton/Dynamics/

La mecánica Hamiltoniana es una reformulación de la mecánica clásica, 
propuesta en 1834 por William R. Hamilton \cite{hamilton1834general}.
Tiene gran importancia en la mecánica cuántica y la mecánica estadística \cite{Montague:399399}.
Su efectividad es observada en el estudio de sistemas con un número
incontable de partículas \cite{morin2008introduction}. 
Cabe resaltar que la mecánica Hamiltoniania considera únicamente a 
sistemas con restricciones holónomas \cite{morin2008introduction}.


\textbf{Restricciones holónomas.} Una restricción $g$ es holónoma si es posible 
expresarla exclusivamente como una ecuación que depende exclusivamente de 
las coordenadas generalizadas $\mathbf q$ del sistema.
\begin{equation}
 g(\mathbf q) = 0
\end{equation}

El péndulo simple posee una restricción holónoma dada por la longitud $l$
y las coordenadas de la masa puntual $m$.

\begin{equation}
 x^2 + y^2 = l^2
 \label{eq: holonomic constraint}
\end{equation}


La formulación Hamiltoniana de la mecánica hace uso de las coordenadas 
generalizadas $\mathbf q = (q_1, \ q_2, \ \dots q_n)^T$, 
los momentos generalizados $\bold p = (p_1 \ p_2 \ \dots p_n)^T$ 
y el Lagrangiano $\mathcal L$ del sistema para describir la energía total del sistema. 
Esta relación se expresa mediante el Hamiltoniano $\mathcal H$. 

\begin{equation}
 \mathcal H = \mathbf p \cdot \mathbf {\dot q} - \mathcal L
 \label{eq: hamiltonian}
\end{equation}

El Hamiltoniano $\mathcal H$ puede ser expresado también como la 
suma de la energía cinética $T$ y la energía  potencial $V$
del sistema.

\begin{equation}
 \mathcal H = T + V
\end{equation}

De acuerdo a \cite{susskind2014theoretical} es posible expresar las velocidades
generalizadas $\mathbf{\dot q}$ y las fuerzas generalizadas $\mathbf{\dot p}$ como 
derivadas parciales de \eqref{eq: hamiltonian}.

\begin{subequations}
 \begin{align}
  \mathbf{\dot p} & = - \dfrac{\partial \mathcal H}{\partial \mathbf q}\\
  \mathbf{\dot q} & = \dfrac{\partial \mathcal H}{\partial \mathbf p}
 \end{align}
\end{subequations}

De \cite{Montague:399399} se establece que el momento generalizado $\mathbf p$
puede ser expresado como la derivada parcial del Lagrangiano $\mathcal L$ respecto a
la velocidad generalizada $\mathbf {\dot q}$.

\begin{equation}
 \mathbf p = \dfrac{\partial \mathcal L}{\partial \mathbf{\dot q}}
 \label{eq: generalized momentum}
\end{equation}








% \subsection{Fricción}
% https://file.scirp.org/pdf/JAMP_2017012515591136.pdf


%  Euler-Lagrange equation


\subsection{Modelo matemático}

Para describir el movimiento de un sistema de péndulo simple 
restringido al plano bidimensional es necesario expresar 
la posición y velocidad de la masa $m$ en función del movimiento
angular $\theta$ y las fuerzas que actúan sobre el sistema.
Se utilizaron dos metodologías para obtener las ecuaiones
de movimiento del sistema: 
\begin{itemize}
 \item Suma de fuerzas del sistema empleando las leyes de movimiento de Newton.
 \item Conservación de energía mediante la ecuación de Euler-Lagrange.
\end{itemize}

%  --------------------------
%  Newton's Laws of Motion
% ---------------------------

Partiendo de las leyes de movimiento de Newton, se establece que la aceleración
$a$ del objeto de masa $m$ puede ser descrita en función de las fuerzas 
presentes en el sistema. 
Como se observa en la figura \ref{fig: pendulum forces}, 
para el caso de un péndulo simple se tienen dos fuerzas:
la fuerza de gravedad $F_{mg}$ que actúa sobre el eje vertical 
y la fuerza de fricción $F_f$ que se opone al movimiento del objeto. 

 \begin{figure}[ht]
    \centering
    \import{./img/}{pendulum_forces.pdf_tex}
    \caption{Diagrama de fuerzas.}
    \label{fig: pendulum forces}
\end{figure}

\begin{equation}
 \begin{split}
  \sum \bold{F} & = m \bold{a}\\
  \sum \bold{F} & = - \bold{F}_{mg} - \bold{F}_f
 \end{split}
 \label{eq: }
\end{equation}




\begin{large}
\begin{gather*}
\sum F = ma = -F_{mg} - F_f \\ \bigskip
ml\ddot{\theta} = -mg\sin(\theta) - kl\dot{\theta} \\ \bigskip
\ddot{\theta} = -\dfrac{g}{l}\sin(\theta) - \dfrac{k}{m}\dot{\theta} \\
\end{gather*} 
\end{large}
\begin{flushright}
\begin{small}
m = masa del péndulo\\
l = largo del péndulo\\
k = constante fricción\\
\end{small}
\end{flushright}

El modelo en base al método de Newton se basa en conocer las fuerzas actuando, las fuerzas principales que actúan sobre el péndulo es la fuerza ocasionada por el peso de la masa ($F_{mg}$) y la fuerza de al fricción que se opone al movimiento del péndulo ($F_f$).\\

El segundo método utilizado para encontrar las ecuaciones de movimiento fue el de energías de Lagrange.
\begin{large}
\begin{equation} \label{L_equ}
\dfrac{d}{dt} \dfrac{\partial L}{\partial \dot{\theta}} - \dfrac{\partial L}{\partial\theta} = 0
\end{equation}
\end{large}
Por este método se necesitan desarrollar las ecuaciones de energía del péndulo.\\
\begin{flushleft}
{\large Energía Cinética}
\end{flushleft}
\begin{equation} \label{T_equ}
T = \frac{1}{2}mv^2 = \frac{1}{2}m(l\theta)^2 = \frac{1}{2}ml^2\theta^2 
\end{equation}
\begin{flushleft}
{\large Energía Potencial}
\end{flushleft}
\begin{equation} \label{V_equ}
V = mgl(1-\cos \theta)
\end{equation}
Con estas ecuaciones se puede definir el Lagrangiano el cual es el que va a ser diferenciado por medio de la ecuación \ref{L_equ}. El Lagrangiano se define como:

\begin{large}
\begin{align*}
L = T - V = \frac{1}{2}m(l\theta)^2 \\
= \frac{1}{2}ml^2\theta^2 - mgl(1-\cos \theta)
\end{align*}
\end{large}

\begin{large}
\begin{equation*}
\dfrac{d}{dt} \dfrac{\partial L}{\partial \dot{\theta}} = \dfrac{d}{dt} ml^2\dot{\theta}
\end{equation*}
\end{large}
\begin{large}
\begin{equation} \label{dLv_equ}
\dfrac{d}{dt} ml^2\dot{\theta} = ml^2\ddot{\theta}
\end{equation}
\end{large}
\begin{large}
\begin{equation} \label{dLp_equ}
\dfrac{\partial L}{\partial\theta} = -gl\sin(\theta)
\end{equation}
\end{large}
Al unir la ecuación \ref{dLv_equ} menos la ecuación \ref{dLp_equ}, en base a la diferenciación del Lagrangiano (ecuación \ref{L_equ}), se obtiene la ecuación de movimiento del sistema.
\begin{large}
\begin{equation}
ml^2\ddot{\theta} + gl\sin(\theta) = 0
\end{equation}
\end{large}

\section{Simulación}

El modelo obtenido en 

\section{Modelo físico}
\subsection{Prueba de concepto}


% \section{Resultados}
En la figura \ref{fig: phase plot x} 
se presenta el diagrama fase del sistema para
posición y velocidad con respecto al eje $x$.




\begin{figure}[h]
 \centering
 \includegraphics[scale=0.3]{./img/tracker_poc_phasediagram_x_vx.png}
 % tracker_poc_phasediagram_x_vx.png: 844x585 px, 72dpi, 29.78x20.64 cm, bb=0 0 844 585
 \caption{Diagrama de fase del modelo físico para $x$ y $\dot{x}$}
 \label{fig: tracker phase diagram x vx}
\end{figure}

\begin{figure}[h]
 \centering
 \includegraphics[scale=0.3]{./img/tracker_poc_phasediagram_y_vy.png}
 % tracker_poc_phasediagram_x_vx.png: 844x585 px, 72dpi, 29.78x20.64 cm, bb=0 0 844 585
 \caption{Diagrama de fase del modelo físico para $y$ y $\dot{y}$}
 \label{fig: tracker phase diagram y vy}
\end{figure}


\begin{figure}[h]
 \centering
 \includegraphics[scale=0.3]{./img/tracker_poc_timeplot_x.png}
 % tracker_poc_phasediagram_x_vx.png: 844x585 px, 72dpi, 29.78x20.64 cm, bb=0 0 844 585
 \caption{Diagrama de tiempo del modelo físico para $x$.}
 \label{fig: tracker time diagram x}
\end{figure}

\section{Conclusiones}
El mecanismo de péndulo simple es ....


\subsection{Natalie}

Comentario feliz.



\clearpage
% \bibliographystyle{IEEEtran}
\printbibliography{}

\pagebreak

\appendix
\section{Código Octave - Newton}
\lstinputlisting[language=Matlab]{../codigosM/octave_pendulum.m}

\section{Código Octave - Hamiltoniano}
\lstinputlisting[language=Matlab]{../codigosM/octave_hamiltonian.m}

\section{Código Octave - Tracker}
\lstinputlisting[language=Matlab]{../codigosM/octave_compare.m}

% \pagebreak
% \lstinputlisting[language=Matlab]{../codigosM/PenduloProy.m}

\end{document}



