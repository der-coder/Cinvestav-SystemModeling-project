\documentclass[conference]{IEEEtran}

\usepackage{hyperref}

\usepackage{cite}
\usepackage{amsmath,amssymb,amsfonts}
\usepackage{algorithmic}
\usepackage{graphicx}
\usepackage{textcomp}
\usepackage{xcolor}
\usepackage[spanish,mexico]{babel}

\usepackage{import}
\usepackage{xifthen}
\usepackage{pdfpages}
\usepackage{transparent}



\def\BibTeX{{\rm B\kern-.05em{\sc i\kern-.025em b}\kern-.08em
    T\kern-.1667em\lower.7ex\hbox{E}\kern-.125emX}}
    
\begin{document}

\title{
Péndulo Simple - Análisis, simulación y construcción\\
}


\author{
    \IEEEauthorblockN{
    Enrique Benavides Téllez, Isaac Ayala Lozano,\\ 
    Sandy Natalie Campos Martínez, Luis Gerardo Almanza Granados\\
    y Yair Casas Flores
    }
    \IEEEauthorblockA{
    \textit{Robótica y Manufactura Avanzada} \\
    \textit{CINVESTAV}\\
        Ramos Arizpe, México}
}

\maketitle

\begin{abstract}

El sistema del péndulo simple es uno de los sistemas más estudiados
en teoría de control. 
Su fácil construcción y modelado permite el diseño de diferentes estrategias
de control y una gran facilidad de probar dichas estrategias antes de 
implementarlas en un sistema más complejo.
El documento presente muestra el estudio del sistema sin linearización.
El estudio comprende la obtención de las ecuaciones de movimiento,
la simulación del mismo en Matlab y la comparación del modelo con
un sistema físico.


\end{abstract}

% % \begin{IEEEkeywords}
% \end{IEEEkeywords}

\section{Introducción}
El mecanismo de péndulo simple es ....


\section{Desarrollo}
\section{Modelo Matemático}


El péndulo simple es un sistema el cual se basa en una particula de masa $m$ sostenido de un punto fijo por medio de una barra o hilo de masa despreciable y sin extenderse mas de su distancia $l$. (Insertar figura del péndulo)\\

Para encontrar el movimiento de un péndulo se utilizaron los métodos de fuerza de Newton y el método de energías de Lagrange. Por medio del método de fuerzas de Newton, se desarrolla de la siguiente manera.
\begin{large}
\begin{gather*}
\sum F = ma = -F_{mg} - F_f \\ \bigskip
ml\ddot{\theta} = -mg\sin(\theta) - kl\dot{\theta} \\ \bigskip
\ddot{\theta} = -\dfrac{g}{l}\sin(\theta) - \dfrac{k}{m}\dot{\theta} \\
\end{gather*} 
\end{large}
\begin{flushright}
\begin{small}
m = masa del péndulo\\
l = largo del péndulo\\
k = constante fricción\\
\end{small}
\end{flushright}

El modelo en base al método de Newton se basa en conocer las fuerzas actuando, las fuerzas principales que actúan sobre el péndulo es la fuerza ocasionada por el peso de la masa ($F_{mg}$) y la fuerza de al fricción que se opone al movimiento del péndulo ($F_f$).\\

El segundo método utilizado para encontrar las ecuaciones de movimiento fue el de energías de Lagrange.
\begin{large}
\begin{equation} \label{L_equ}
\dfrac{d}{dt} \dfrac{\partial L}{\partial \dot{\theta}} - \dfrac{\partial L}{\partial\theta} = 0
\end{equation}
\end{large}
Por este método se necesitan desarrollar las ecuaciones de energía del péndulo. Y para desarrollar las ecuaciones de posición asignamos el marco de referencia del péndulo y se obtiene:\\
\begin{equation}
\left(\begin{matrix}
x\\
y
\end{matrix}\right) = 
\begin{bmatrix}
l\sen(\theta)\\
l(1 - \cos(\theta)
\end{bmatrix}
\end{equation}

\begin{flushleft}
{\large Energía Cinética}
\end{flushleft}
\begin{equation} \label{T_equ}
T = \frac{1}{2}mv^2 = \frac{1}{2}m(\dot{x}^2 + \dot{y}^2) 
\end{equation}
Derivando las ecuaciones de posición del péndulo obtenemos las ecuaciones de velocidad:
\begin{equation} \label{Vel_p}
\left(\begin{matrix}
\dot{x}\\
\dot{y}
\end{matrix}\right) = 
\begin{bmatrix}
l\dot{\theta}\cos(\theta)\\
l\dot{\theta}\sen(\theta)
\end{bmatrix}
\end{equation}
Sustituyendo la ecuación \ref{Vel_p} en \ref{T_equ} y desarrollando se obtiene:
\begin{equation}
T = \frac{1}{2}ml^2\dot{\theta}^2
\end{equation}
\begin{flushleft}
{\large Energía Potencial}
\end{flushleft}
La energía potencial se plantea multiplicando la posición en el \emph{eje y} del péndulo por la masa y gravedad. Se plantea de la siguiente manera:
\begin{equation} \label{V_equ}
V = mgl(1-\cos \theta)
\end{equation}
Con estas ecuaciones se puede definir el Lagrangiano el cual es el que va a ser diferenciado por medio de la ecuación \ref{L_equ}. El Lagrangiano se define como:

\begin{large}
\begin{align*}
L = T - V = \frac{1}{2}m(l\theta)^2 \\
= \frac{1}{2}ml^2\dot{\theta}^2 - mgl(1-\cos \theta)
\end{align*}
\end{large}

\begin{large}
\begin{equation*}
\dfrac{d}{dt} \dfrac{\partial L}{\partial \dot{\theta}} = \dfrac{d}{dt} ml^2\dot{\theta}
\end{equation*}
\end{large}
\begin{large}
\begin{equation} \label{dLv_equ}
\dfrac{d}{dt} ml^2\dot{\theta} = ml^2\ddot{\theta}
\end{equation}
\end{large}
\begin{large}
\begin{equation} \label{dLp_equ}
\dfrac{\partial L}{\partial\theta} = -gl\sin(\theta)
\end{equation}
\end{large}
Al unir la ecuación \ref{dLv_equ} menos la ecuación \ref{dLp_equ}, en base a la diferenciación del Lagrangiano (ecuación \ref{L_equ}), se obtiene la ecuación de movimiento del sistema.
\begin{large}
\begin{equation}
ml^2\ddot{\theta} + gl\sin(\theta) = 0
\end{equation}
\end{large}


\section{Resultados}
El mecanismo de péndulo simple es ....


\section{Conclusiones}
\section{Conclusiones}
El proyecto concluye de manera exitosa.
La implementación de un modelo no lineal de péndulo simple y la construcción
de un modelo para validar los resultados producidos 
fue una experiencia bastante enriquecedora.
Esto debido al alcance total del proyecto: modelado,
simulación, construcción, calibración y comparación.\\

Cada etapa del proyecto requirió de habilidades diferentes, así como 
de la búsqueda de información adecuada para poder avanzar con el desarrollo.
La etapa de modelado hizo uso de las distintas formulaciones de mecánica 
que fueron presentadas durante el curso.
El desarrollo del simulador a su vez requirió de habilidades de programación
 y la capacidad de desarrollar una estrategia de trabajo que permitiera integrar
 los resultados de simulación con las áreas subsecuentes: documentación y construcción.\\

Mediante el uso de múltiples herramientas de medición fue posible calibrar el 
simulador para que el modelo virtual describa un movimiento mucho 
más cercano al comportamiento del péndulo real.
Además, la integración de nuevas herramientas de análisis permitió determinar 
propiedades del sistema que no eran posibles con los sensores disponibles.
El uso de Tracker demostró ser esencial para la obtención de los nuevos valores
del sistema debido a su capacidad de obtener detalles de la mecánica del sistema
que no fueron posibles medir con las herramientas disponibles.\\

Finalmente, el planteamiento del sistema bajo las distintas formulaciones 
de mecánica permiten obtener nuevas perspectivas sobre el modelado de sistemas.
Las formulaciones basadas en energía demuestran su poder ser adaptadas
de manera más fácil a cualquier conjunto de coordenadas.
Esta ventaja sobre la formulación de Newton resulta atractiva para modelos más complejos.
No obstante es esencial tomar en cuenta qué restricciones tienen estas formulaciones
y si es posible extenderlas para incluir los efectos de fuerzas no conservativas.


here \cite{susskind}.

\bibliographystyle{IEEEtran}
\bibliography{bibliografia.bib}
\end{document}
