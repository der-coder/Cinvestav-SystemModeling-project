\documentclass[journal, onecolumn]{IEEEtran}

\usepackage{hyperref}

\usepackage{cite}
\usepackage{amsmath,amssymb,amsfonts}
\usepackage{algorithmic}
\usepackage{graphicx}
\usepackage{textcomp}
\usepackage{xcolor}
\usepackage[spanish,mexico]{babel}

\usepackage{import}
\usepackage{xifthen}
\usepackage{pdfpages}
\usepackage{transparent}

\usepackage{bm}



\def\BibTeX{{\rm B\kern-.05em{\sc i\kern-.025em b}\kern-.08em
    T\kern-.1667em\lower.7ex\hbox{E}\kern-.125emX}}
    
    \DeclareRobustCommand{\uvec}[1]{{%
  \ifcat\relax\noexpand#1%
    % it should be a Greek letter
    \bm{\hat{#1}}%
  \else
    \ifcsname uvec#1\endcsname
      \csname uvec#1\endcsname
    \else
      \bm{\hat{\mathbf{#1}}}%
     \fi
   \fi
}}
    
\begin{document}

\title{
Péndulo Simple - Análisis, simulación y construcción\\
}


\author{
    \IEEEauthorblockN{
    Enrique Benavides Téllez, Isaac Ayala Lozano,\\ 
    Sandy Natalie Campos Martínez, Luis Gerardo Almanza Granados\\
    y Yair Casas Flores\\
    }
    \IEEEauthorblockA{
    \textit{Robótica y Manufactura Avanzada} \\
    \textit{CINVESTAV}\\
        Ramos Arizpe, México}
}

\maketitle

\begin{abstract}

El sistema del péndulo simple es uno de los sistemas más estudiados
en teoría de control. 
Su diseño y contrucción de baja dficultad hacen de este sistema uno de
los más accesibles para modelar y contrastar con una implementación
física.
El documento presente muestra el estudio del sistema sin linearización.
El estudio comprende la obtención de las ecuaciones de movimiento,
la simulación del mismo en MATLAB y la comparación del modelo con
un sistema físico.


\end{abstract}

% % \begin{IEEEkeywords}
% \end{IEEEkeywords}

\section{Introducción}
\section{Introducción}

El péndulo simple ha sido uno de los mecanismos más estudiados 
por la comunidad científica a lo largo del tiempo, 
y sus aplicaciones han sido vastas. 
Galileo Galilei describió el comportamiento de
este mecanismo en 1602 \cite{drake2003galileo}, 
llegando a la conclusión de que el movimiento del mismo es 
\emph{isócrono}.
Quizás la aplicación más importante del péndulo simple ha sido
la invención del físico y matemático Christian Huygens: el reloj
de péndulo.
Inventado en 1656 y patentado en 1657 \cite{bennet2002huygenclock}, 
el reloj de péndulo demostró ser el instrumento más preciso para
la medición del tiempo hasta la construcción del reloj de
cuarzo en 1927 en Bell Laboratories \cite{morrison1948quartzcrystalclock}.\\

Es evidente que la comprensión detallada de este mecanismo
ha dado lugar a avances científicos e invenciones importantes 
a lo largo de la historia de la humanidad.
Este trabajo pretende mostrar un análisis de dicho mecanismo
empleando dos metodologías para su análisis.
El trabajo se enfoca en tres áreas de interés para el estudio
del péndulo simple: el modelo matemático del sistema, 
la simulación del modelo matemático empleando 
herramientas computacionales y una implementación
física del mecanismo.


\section{Modelo Matemático}

\subsection{Modelo matemático}

Para describir el movimiento de un sistema de péndulo simple 
restringido al plano bidimensional es necesario expresar 
la posición y velocidad de la masa $m$ en función del movimiento
angular $\theta$ y las fuerzas que actúan sobre el sistema.
Se utilizaron dos metodologías para obtener las ecuaiones
de movimiento del sistema: 
\begin{itemize}
 \item Suma de fuerzas del sistema empleando las leyes de movimiento de Newton.
 \item Conservación de energía mediante la ecuación de Euler-Lagrange.
\end{itemize}

%  --------------------------
%  Newton's Laws of Motion
% ---------------------------

Partiendo de las leyes de movimiento de Newton, se establece que la aceleración
$a$ del objeto de masa $m$ puede ser descrita en función de las fuerzas 
presentes en el sistema. 
Como se observa en la figura \ref{fig: pendulum forces}, 
para el caso de un péndulo simple se tienen dos fuerzas:
la fuerza de gravedad $F_{mg}$ que actúa sobre el eje vertical 
y la fuerza de fricción $F_f$ que se opone al movimiento del objeto. 

 \begin{figure}[ht]
    \centering
    \import{./img/}{pendulum_forces.pdf_tex}
    \caption{Diagrama de fuerzas.}
    \label{fig: pendulum forces}
\end{figure}

\begin{equation}
 \begin{split}
  \sum \bold{F} & = m \bold{a}\\
  \sum \bold{F} & = - \bold{F}_{mg} - \bold{F}_f
 \end{split}
 \label{eq: }
\end{equation}




\begin{large}
\begin{gather*}
\sum F = ma = -F_{mg} - F_f \\ \bigskip
ml\ddot{\theta} = -mg\sin(\theta) - kl\dot{\theta} \\ \bigskip
\ddot{\theta} = -\dfrac{g}{l}\sin(\theta) - \dfrac{k}{m}\dot{\theta} \\
\end{gather*} 
\end{large}
\begin{flushright}
\begin{small}
m = masa del péndulo\\
l = largo del péndulo\\
k = constante fricción\\
\end{small}
\end{flushright}

El modelo en base al método de Newton se basa en conocer las fuerzas actuando, las fuerzas principales que actúan sobre el péndulo es la fuerza ocasionada por el peso de la masa ($F_{mg}$) y la fuerza de al fricción que se opone al movimiento del péndulo ($F_f$).\\

El segundo método utilizado para encontrar las ecuaciones de movimiento fue el de energías de Lagrange.
\begin{large}
\begin{equation} \label{L_equ}
\dfrac{d}{dt} \dfrac{\partial L}{\partial \dot{\theta}} - \dfrac{\partial L}{\partial\theta} = 0
\end{equation}
\end{large}
Por este método se necesitan desarrollar las ecuaciones de energía del péndulo.\\
\begin{flushleft}
{\large Energía Cinética}
\end{flushleft}
\begin{equation} \label{T_equ}
T = \frac{1}{2}mv^2 = \frac{1}{2}m(l\theta)^2 = \frac{1}{2}ml^2\theta^2 
\end{equation}
\begin{flushleft}
{\large Energía Potencial}
\end{flushleft}
\begin{equation} \label{V_equ}
V = mgl(1-\cos \theta)
\end{equation}
Con estas ecuaciones se puede definir el Lagrangiano el cual es el que va a ser diferenciado por medio de la ecuación \ref{L_equ}. El Lagrangiano se define como:

\begin{large}
\begin{align*}
L = T - V = \frac{1}{2}m(l\theta)^2 \\
= \frac{1}{2}ml^2\theta^2 - mgl(1-\cos \theta)
\end{align*}
\end{large}

\begin{large}
\begin{equation*}
\dfrac{d}{dt} \dfrac{\partial L}{\partial \dot{\theta}} = \dfrac{d}{dt} ml^2\dot{\theta}
\end{equation*}
\end{large}
\begin{large}
\begin{equation} \label{dLv_equ}
\dfrac{d}{dt} ml^2\dot{\theta} = ml^2\ddot{\theta}
\end{equation}
\end{large}
\begin{large}
\begin{equation} \label{dLp_equ}
\dfrac{\partial L}{\partial\theta} = -gl\sin(\theta)
\end{equation}
\end{large}
Al unir la ecuación \ref{dLv_equ} menos la ecuación \ref{dLp_equ}, en base a la diferenciación del Lagrangiano (ecuación \ref{L_equ}), se obtiene la ecuación de movimiento del sistema.
\begin{large}
\begin{equation}
ml^2\ddot{\theta} + gl\sin(\theta) = 0
\end{equation}
\end{large}


\section{Simulación}
\section{Simulación}

El modelo obtenido en 


\section{Resultados}
\section{Resultados}
En la figura \ref{fig: phase plot x} 
se presenta el diagrama fase del sistema para
posición y velocidad con respecto al eje $x$.




\begin{figure}[h]
 \centering
 \includegraphics[scale=0.3]{./img/tracker_poc_phasediagram_x_vx.png}
 % tracker_poc_phasediagram_x_vx.png: 844x585 px, 72dpi, 29.78x20.64 cm, bb=0 0 844 585
 \caption{Diagrama de fase del modelo físico para $x$ y $\dot{x}$}
 \label{fig: tracker phase diagram x vx}
\end{figure}

\begin{figure}[h]
 \centering
 \includegraphics[scale=0.3]{./img/tracker_poc_phasediagram_y_vy.png}
 % tracker_poc_phasediagram_x_vx.png: 844x585 px, 72dpi, 29.78x20.64 cm, bb=0 0 844 585
 \caption{Diagrama de fase del modelo físico para $y$ y $\dot{y}$}
 \label{fig: tracker phase diagram y vy}
\end{figure}


\begin{figure}[h]
 \centering
 \includegraphics[scale=0.3]{./img/tracker_poc_timeplot_x.png}
 % tracker_poc_phasediagram_x_vx.png: 844x585 px, 72dpi, 29.78x20.64 cm, bb=0 0 844 585
 \caption{Diagrama de tiempo del modelo físico para $x$.}
 \label{fig: tracker time diagram x}
\end{figure}


\section{Conclusiones}
\section{Conclusiones}
El mecanismo de péndulo simple es ....


\subsection{Natalie}

Comentario feliz.


here \cite{susskind}.

\bibliographystyle{IEEEtran}
\bibliography{bibliografia.bib}
\end{document}
